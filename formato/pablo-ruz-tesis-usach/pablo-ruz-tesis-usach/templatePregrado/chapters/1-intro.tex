\chapter{Introducci\'on}
\label{cap:intro}

\section{Antecedentes y motivaci\'on}
\label{intro:motivacion}

 \ldots 


\section{Descripci\'on del problema}
\label{intro:problema}

%La definición de claves XML es más compleja que en el modelo %relacional, debido a la compleja estructura de árbol que poseen los %documentos XML.

%En este trabajo se plantea, en primer lugar, determinar la utilidad de trabajar en la práctica con claves XML utilizando un algoritmo para el problema de implicación. Definir ésta utilidad práctica permitirá avanzar en la aceptación de las claves como restricciones sobre XML por parte de los profesionales, considerando el poder expresivo que estas entregan a XML. En segundo lugar, existe la necesidad de un método que permita  determinar la validez de un documento XML contra un conjunto predefinido de claves XML. A partir de los trabajos realizados en validación de documentos XML contra claves \citep{Abrao:2004, Bouchou:2003, Chen:2002, Liu:2005, Liu:2004}, se plantea diseñar un algoritmo que permita validar documentos XML contra claves XML como las definidas en \citet{Buneman:2003,Buneman:2002}, las cuales consideran la igualdad en valor entre nodos elemento: si los subárboles que tienen por raíz a estos nodos, son isomorfos por algún isomorfismo que para cadenas de texto se corresponde con la función 
%identidad.

%Finalmente, considerando que la complejidad del algoritmo de validación depende en parte del tamaño del conjunto de claves, se investiga un método para obtener una optimización del proceso de validación de documentos XML contra claves, utilizando el algoritmo de implicación de claves XML presentado por \citet{HartmannLink:2009}.

\section{Soluci\'on propuesta}
\label{intro:solucion}


\section{Objetivos y alcance del proyecto}
\label{intro:objetivos}

\subsection{Objetivo general}

Diseñar una aplicación de gestión de la producción y el mantenimiento correctivo y preventivo para la optimización de procesos de impresión 3D FDM.

\subsection{Objetivos espec\'ificos}

Para la consecución del objetivo general, se plantean las siguientes metas intermedias:

\begin{enumerate}
 
	\item Determinar las variables implicadas en el proceso que permiten obtener indicadores.
	\item Investigar compatibilidad entre hardware, software, protocolos de comunicación, y códigos de programación a utilizar.
	\item Elaborar registros y fichas técnicas de impresoras 3D.
	\item Establecer relaciones matemáticas que permitan entregar indicadores relevantes para la producción y mantenimiento.
	\item Diseñar funciones que permitan gestionar los datos de hardware y software para determinación de indicadores.
	\item Diseñar interfaz de aplicación orientado al usuario. 
	
\end{enumerate} \ldots 
\subsection{Alcances}

Se pretende desarrollar una Interfaz Programable de Aplicación utilizando como base el software Octoprint, pudiendo controlar, monitorizar en tiempo real el funcionamiento de varias impresoras 3D, y entregar indicadores para gestionar la producción y el mantenimiento de las máquinas. Para esto, se toman en cuenta los siguientes alcances:

\begin{enumerate}
	\item Emplear metodologías ágiles para el diseño.
	\item Utilizar softwares y herramientas de código abierto.
	\item Trabajar en una plataforma cliente/servidor.
	\item Diseñar un sistema enfocado en el usuario.
	\item Tomar las entradas de impresoras, lista de piezas, tiempos de producción, peso de filamento y tiempo de actividad. 
	\item Configurar planificación y frecuencia de mantenimientos autónomos y preventivos.
	\item Configurar planificación y emitir órdenes de producción.
	\item Emitir reportes y consultas sobre el estado de las órdenes de producción y mantenimiento. 


\end{enumerate}  

\section{Metodolog\'ia y herramientas utilizadas}
\label{intro:metodologia}

\subsection{Metodolog\'ia}


\subsection{Herramientas de desarrollo}


\section{Resultados Obtenidos}
\label{intro:resultados}


\section{Organizaci\'on del documento}
\label{intro:organizacion}

El presente trabajo está dividido en ocho capítulos considerando éste como el primero. En el Capítulo~\ref{cap:preliminares} se formalizan los fundamentos de documento XML, modelo de árbol XML, y lenguaje de definición de expresiones de camino para definir claves XML. \ldots
