\chapter{Introducci\'on}
\label{cap:intro}

\section{Antecedentes y motivaci\'on}
\label{intro:motivacion}

 \ldots 


\section{Descripci\'on del problema}
\label{intro:problema}



\section{Soluci\'on propuesta}
\label{intro:solucion}


\section{Objetivos y alcance del proyecto}
\label{intro:objetivos}

\subsection{Objetivo general}

Diseñar una aplicación de gestión de la producción y el mantenimiento correctivo y preventivo para la optimización de procesos de impresión 3D FDM.

\subsection{Objetivos espec\'ificos}

Para la consecución del objetivo general, se plantean las siguientes metas intermedias:

\begin{enumerate}
 
	\item Determinar las variables implicadas en el proceso que permiten obtener indicadores.
	\item Investigar compatibilidad entre hardware, software, protocolos de comunicación, y códigos de programación a utilizar.
	\item Elaborar registros y fichas técnicas de impresoras 3D.
	\item Establecer relaciones matemáticas que permitan entregar indicadores relevantes para la producción y mantenimiento.
	\item Diseñar funciones que permitan gestionar los datos de hardware y software para determinación de indicadores.
	\item Diseñar interfaz de aplicación orientado al usuario. 
	
\end{enumerate} \ldots 
\subsection{Alcances}

Se pretende desarrollar una Interfaz Programable de Aplicación utilizando como base el software Octoprint, pudiendo controlar, monitorizar en tiempo real el funcionamiento de varias impresoras 3D, y entregar indicadores para gestionar la producción y el mantenimiento de las máquinas. Para esto, se toman en cuenta los siguientes alcances:

\begin{enumerate}
	\item Emplear metodologías ágiles para el diseño.
	\item Utilizar softwares y herramientas de código abierto.
	\item Trabajar en una plataforma cliente/servidor.
	\item Diseñar un sistema enfocado en el usuario.
	\item Tomar las entradas de impresoras, lista de piezas, tiempos de producción, peso de filamento y tiempo de actividad. 
	\item Configurar planificación y frecuencia de mantenimientos autónomos y preventivos.
	\item Configurar planificación y emitir órdenes de producción.
	\item Emitir reportes y consultas sobre el estado de las órdenes de producción y mantenimiento. 


\end{enumerate}  

\section{Metodolog\'ia y herramientas utilizadas}
\label{intro:metodologia}

\subsection{Metodolog\'ia}


\subsection{Herramientas de desarrollo}


\section{Resultados Obtenidos}
\label{intro:resultados}


\section{Organizaci\'on del documento}
\label{intro:organizacion}


