\chapter{Conclusiones}
\label{cap:conc}

En términos generales, se puede decir que se han cumplido los objetivos propuestos para el presente trabajo de título, relacionados al diseño de una aplicación que sirva de asistente para la gestión de la producción y el mantenimiento correctivo y preventivo de impresoras 3D FDM.  Se debe mencionar que, durante todo el proceso de mantenimiento basado en la confiabilidad se obvía la impresora \textit{Ender 3}, debido a que fue adquirida y puesta en marcha durante la realización de este trabajo, por tanto el registro de fallas es mínimo y no se pueden establecer conclusiones estadísticas con tan pocos datos. En primer lugar, se evalúa la criticidad de los equipos de la granja de impresión 3D realizando el análisis de falla según el diagrama de Pareto para tres de las cuatro impresoras del taller. Debido a que esta técnica pone énfasis en una sola variable, y como la cantidad de máquinas se limita a cuatro impresoras, se utiliza para identificar qué fallas son las que más se repiten en cada equipo. No obstante, también se puede utilizar para encontrar aquellas impresoras que requieren mayor tiempo de reparación, puesto que el levantamiento de información considera los tiempos aproximados de reparación. La aplicación de esta herramienta muestra que la mayor frecuencia de fallas se encuentra en la extrusión deficiente o nula de filamento, seguido de la adhesión de las piezas a la superficie de impresión. 
Para elaborar el análisis de criticidad se evalúan aspectos como el impacto a la seguridad, el costo de reparación, la frecuencia de las fallas y el costo operacional implicado. Según las ponderaciones propuestas, se obtiene que la impresora X350 German RepRap es el equipo con mayor criticidad, según el periodo comprendido entre mayo y octubre de 2020. En función de este resultado, se caracteriza y elabora la ficha técnica de la máquina, explicitando sus componentes. Luego, se delimita el contexto operacional del funcionamiento de la granja de impresión, tanto la capacidad instalada como los factores climáticos que interceden en la producción.
La identificación de las conexiones lógicas o secuenciales del funcionamiento de una impresora 3D se realiza a través de un diagrama de interrelaciones funcionales, donde se identifican a lo menos cinco subsistemas: alimentación, control, construcción, movimiento y extrusión, todos con sus respectivas entradas y salidas. Se debe notar que, cada interrupción lógica entre estos subsistemas implica una falla en el sistema de la impresora, por tanto, para determinar estas fallas se realiza el método de árbol de fallas. Esta información da a lugar a la realización de un análisis de modos y efectos de falla, el cual permitió una descripción detallada del equipo, tanto en sus componentes, funciones, modos y efectos de las fallas. Los resultados obtenidos de este análisis están directamente relacionados con la creación e implementación de los planes de mantenimiento, y son el punto de partida para establecer las planillas de decisión a partir del diagrama del mismo nombre. Las decisiones respecto a las fallas establecen mantenimientos autónomos centrados en la inspección diaria de la máquina, mantenimientos preventivos semanales y mensuales, y mantenimientos a condición.
Consecutivamente, se estudia la confiabilidad de los equipos del taller a través de un análisis de Weibull, el cual considera el historial de fallas de cada equipo. El análisis se realiza por medio de dos técnicas contrastadas, la linealización de la función de distribución acumulativa, y la utilización de librerías de análisis matemático en lenguaje \textit{python}. Los parámetros obtenidos entregan información respecto a la vida estimada del elemento en estudio, como la duración para el 63,2\% de los casos. Se debe destacar que la aplicación correcta del análisis de Weibull está enfocada en las fallas de cada subsistema o componente de una máquina, dado que, por ejemplo, la esperanza de vida de componentes como una boquilla no es la misma que los rodamientos lineales del conjunto extrusor; asimismo, la cantidad de datos obtenidos en el levantamiento de datos no entrega la información necesaria, puestos que son insuficientes para un análisis de este tipo. No obstante, el resultado obtenido por medio de este análisis si puede ser útil para estimar la confiabilidad y la tasa de fallos del sistema en general, y por tanto, permite ajustar la frecuencia de las revisiones, inspecciones y planes de mantenimiento. 
La elaboración del plan de mantenimiento se separa en inspección diaria/proactiva, y mantenimientos preventivos semanales, mensuales y bimensuales. La distinción entre estas actividades se desprenden tanto del análisis de modo y efectos de falla, como del análisis de confiabilidad. En la misma línea, como este trabajo se enfoca en el diseño de una aplicación web, se busca registrar la mayor cantidad de información disminuyendo la interacción entre el usuario y el ingreso de datos, puesto que su mala utilización puede falsear o volver inutilizable la información respecto al mantenimiento y las fallas. 
Una vez realizado el mantenimiento basado en la teoría de la confiabilidad, se aplica la metodología de Design Thinking para el desarrollo de la aplicación web. Se elige esta metodología puesto que permite situar al operador de una granja de impresión 3D en el centro del desarrollo, en función de la mejora de un proceso conocido a través de una solución innovadora y enfocada en usuario. Al plantear la situación actual, se observa que una aplicación para el control y monitoreo de impresoras puede reducir el tiempo y las decisiones que el operario debe realizar para imprimir una pieza y verificar el estado de la producción, ya sea reduciendo los escenarios u optimizando la cantidad de ventanas abiertas en el computador. Dentro de las fases de Design Thinking, el empatizar permite obtener los beneficiarios directos o indirectos de la solución propuesta, teniendo como centro al usuario operario; en el mismo sentido, el mapa de experiencias en etapas definidas da como resultado oportunidades de mejora respecto a la valoración que el usuario otorga a cada paso que debe dar para imprimir una pieza. En lo que respecta a la investigación se pudo determinar que, en lo relacionado a los referentes, no existe una aplicación o software enfocada en el mantenimiento de impresoras, sino algunas implementaciones que, de todas formas, implican que toda la obtención de datos se realiza por medio del administrador de la plataforma de mantenimiento o producción. Por tanto, la recopilación de datos por medio de la automatización de consultas a la API de Octoprint resulta un criterio de innovación con respecto a los referentes estudiados. Gracias a la amplitud de las consultas posibles y disponibles en la documentación de Octoprint, se determina que las variables implicadas en el proceso y que permiten obtener indicadores pueden ser obtenidas a través de la programación de la aplicación web, dentro de los que se encuentran el estado de las piezas en impresión, estado de la impresora, progreso de la operación, piezas canceladas, piezas terminadas, tiempo de impresión y volumen de filamento. Todas estas variables, presentadas al usuario y contrastadas entre sí, le permiten tomar mejores decisiones respecto a si se necesitan realizar ajustes al proceso de configuración de una impresión, adquirir repuestos en el mediano plazo, comprar nuevos filamentos, entre otras.
Por otra parte, se verifica la compatibilidad entre el hardware, software y protocolos de comunicación. Esto pues Octoprint puede ser fácilmente instalada en la placa Raspberry Pi, así también el servidor de programación de flujos Node-Red, la interfaz Grafana, el servidor Flask basado en python, y la plataforma CRUD de bases de datos. La estructura de las bases de datos y su exportación a formato csv permitió comprender que las variables implicadas en el registro de mantenimientos correctivos y acciones no planificadas corresponden a la marca temporal en la que se realiza un mantenimiento, el sistema donde ocurre y su estado de inicio o término. Así, el tiempo medio entre fallos se obtiene como la división de la diferencia de las marcas temporales de dos inicios de mantenimiento correctivo consecutivos, entre la cantidad de fallos ocurridos. En la misma linea, el tiempo medio para reparar resulta del cociente entre la diferencia de marcas temporales de inicio y término de un mismo evento, y la cantidad de fallos ocurridos. Notar que, para que esta última relación matemática tenga una interpretación correcta, un evento de mantenimiento correctivo debe estar obligatoriamente terminado, de forma que la resta entre las dos marcas temporales sea efectiva; en caso contrario, la opearación matemática tiene como minuendo una cantidad nula. Pues bien, la utilización de Node-red permite realizar funciones que cumplan esa condición, haciendo confiable el cálculo y su resultado. Siguiendo con el establecimiento de relaciones matemáticas, la confiabilidad se obtiene como una función entre el tiempo medio entre fallas y el tiempo medio para reparar, y su resultado se representa como un porcentaje. Este indicador se considera importante en una granja de impresoras 3D, en la medida que permite al operario priorizar impresoras para una producción específica o elegir las que tengan mejor confiabilidad para producciones de alto volumen. Las funciones que permiten la gestión de la data y su transformación en información se realiza a través de Node-red, en simultáneo con python. Ambos lenguajes y/o paradigmas de programación se ven unidos a través de las bases de datos Mysql, y las bases de datos de series de tiempo Influxdb. Se debe mencionar que todas las herramientas de programación utilizadas en este trabajo son de código abierto, donde su código fuente está disponible en repositorios de internet. 
Respecto a orientaciones futuras se distinguen, a lo menos, tres categorías para el mejoramiento y desarrollo del proyecto. La primera, y desde el punto de vista del mantenimiento y la obtención de data, corresponde a la investigación y utilización de nuevos sensores, como por ejemplo sensores de corriente, voltaje, o temperatura. Con estas nuevas mediciones, se puede obtener estadísticas precisas respecto a la energía eléctrica consumida por la impresora, o indicadores relativos al mantenimiento y limpieza del gabinete de la placa controladora. De la misma forma, el aumentar la cantidad de periféricos y variables a medir, puede facilitar tanto la toma de decisiones como el control de la impresora, ya sea al incluir una cámara remota, o un relé que active el encendido de la máquina. Otra área de mejora, y en función de los parámetros de diseño, corresponde a la obtención de un producto mínimo viable de la aplicación desarrollada en este proyecto. Para esto, es necesario realizar una nueva iteración según la metodología de Design Thinking, donde se planteen nuevas estrategías para obtener parámetros de mejora, tanto en la interfaz como en la lógica del programa. Finalmente, la última área de mejora corresponde a la escalabilidad y aseguramiento de la calidad del software. En esta parte está involucrada el desarrollo de nuevos códigos orientados a la seguridad de la aplicación, resolución de problemas no detectados en la fase de testeo, y asegurar que a medida que crezcan los recursos utilizados por el programa, así también lo haga su capacidad de procesamiento. Para esto, se debe evaluar si la arquitectura actual es idónea para realizar de buena manera el aumento del hardware o conexiones realizadas al servidor.