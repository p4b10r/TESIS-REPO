
\resumenCastellano{
El presente trabajo consiste en el desarrollo de una aplicación web para la gestión de la producción y el mantenimiento de impresoras 3D utilizando el software Octoprint, aplicado en la empresa de fabricación digital 3Dlux. 
Para la ejecución del proyecto, se realiza un levantamiento de información correspondiente al historial de fallas para cada máquina dispuesta en el taller. La realización del plan de mantenimiento se basa en los resultados obtenidos aplicando la teoría de la confiabilidad, criticidad de los equipos, la delimitación del contexto operacional y funciones, diagrama de sistemas y confección de los árboles de falla. Posteriormente, el análisis de modos y efectos de falla permite dar una estructura del análisis, así tambien los diagramas de decisión para las acciones de mantenimiento a realizar. Para la determinación de los parámetros de forma y escala, se efectúa un análisis de Weibull para cada impresora en cuestión. Seguidamente, la aplicación de la metodología de Design Thinking posibilita un desarrollo centrado en el usuario operador, considerando la situación actual, la investigación de referentes y herramientas de desarrollo. El resultado consiste en una aplicación web para el registro y control de mantenimientos autónomos, proactivos y preventivos, registro de mantenimientos correctivos y no planificados, y el monitoreo de la condición a partir del estado en vivo de la máquina e indicadores de confiabilidad.
\vspace*{0.5cm}
\KeywordsES{Impresión 3D; Mantenimiento; Confiabilidad; Aplicación Web; Design Thinking}
}


