


\section{Temario}
\label{temario}


\subsection{Capítulo I Descripción del proyecto}
		\begin{enumerate}
			\item Objeto de estudio
			\item Planteamiento del problema
			\item Objetivos
				\begin{enumerate}
					\item Objetivo General
					\item Objetivos específicos 
				\end{enumerate}		
		\end{enumerate}
	
\subsection{Capítulo II Marco teórico}

		\begin{enumerate}
			\item Impresora 3D
				\begin{enumerate}
					\item Historia de la impresión 3D	
					\item Métodos de impresión 3D
					\item Impresoras 3D FDM
					\item Tipologías de impresoras 3D FDM 
					\item Componentes de Impresoras 3D FDM	
				\end{enumerate}
			\item Mantenimiento
				\begin{enumerate}
					\item Historia y evolución del mantenimiento
					\item Tipos de mantenimiento
					\item GMAO 
					\item PAS 55 e ISO 55000
					\clearpage
				\end{enumerate}					
			\item Lean Manufacturing
				\begin{enumerate}
					\item Historia Lean Manufacturing
					\item Herramientas de mantenimiento
				\end{enumerate}
			\item Design Thinking y Scrum
				\begin{enumerate}
					\item Metodologías ágiles 
					\item Scrum 					
					\item Design Thinking
					\item Fases del Design Thinking
					\item Herramientas para el diseño de software
				\end{enumerate}
			\item Desarrollo de Software
				\begin{enumerate}	
					\item Programación orientada a objetos
					\item Python
					\item HTML
					\item CSS
					\item JSON
					\item SQL
					\item API  
					\item Arquitectura Cliente-Servidor
					\item Ordenadores de placa reducida 
				\end{enumerate}						
		\end{enumerate}
		\clearpage
\subsection{Capítulo III Estado del Arte}
	
		\begin{enumerate}
			\item Software GMAO
			\item Software Gestión de la impresión 3D
			\item Design Thinking enfocado en Software
		\end{enumerate}
	
\subsection{Capítulo IV Desarrollo del proyecto}

		\begin{enumerate}
			\item Descripción de la Empresa 
			\item Descripción del Problema 
			\item Aplicación de Design Thinking para el desarrollo de Software con metodologías ágiles
				\begin{enumerate}
					\item Toma de requerimientos de usuario
					\item Desarrollo de iteraciones
					\item Verificación y validación de usuario.		
				\end{enumerate}		
		\end{enumerate}
		
		
\subsection{Capítulo V Resultados}
		
	
		\begin{enumerate}
			\item Situación actual de la empresa 
			\item Realización de pruebas
			\item Definición de indicadores
			\item Toma y comparativa de métricas 
			\item Elaboración de documentación 
		
		\end{enumerate}
		
\subsection{Capítulo VI Conclusiones y observaciones}

\subsection{Bibliografía}
\subsection{Anexos}
\clearpage	
	

