\section{Resumen}
\label{intro:resumen}

Si bien las tecnologías aditivas y procesos de impresión 3D datan de la década de los ochenta, la internalización latente en el mercado mundial y el bajo costo que alcanzan las máquinas acompañadas de buenas calidades superficiales, han hecho que la impresión 3D salga del uso asociado al prototipado rápido y son cada vez más usadas para la producción en masa. En este sentido, y para obtener procesos confiables y productivos, es necesaria la determinación de ciertas variables y acciones asociadas a las máquinas, información que actualmente es recogida por los principales softwares de Manufactura Asistida por Computadora, mas no es utilizada o aprovechada de forma idónea. Por otra parte, existen herramientas como el software \textit{Octoprint}, el cual cumple la función de controlar una impresora 3D desde un servidor, lo que implica una gran ventaja en comparación con la acción manual, permitiendo el control remoto desde cualquier parte con acceso a Internet; no obstante, la aplicación solo funciona en una sola impresora, limitando su funcionalidad. Asimismo, el servidor solo realiza actividades de monitoreo y control de la máquina, dejando de lado acciones relativas a la producción misma de piezas o el mantenimiento de las máquinas.       
El presente trabajo consiste en el desarrollo de una aplicación web API para la gestión de la producción y el mantenimiento de múltiples impresoras 3D FDM, por medio de la utilización del servidor e interfaz web \textit{Octoprint}, utilizando el ordenador de placa reducida \textit{Raspberry Pi}.  
En primer lugar, se determina el objeto de estudio y el planteamiento del problema a solucionar, seguido de los objetivos generales y específicos de este trabajo. Seguidamente, a modo de marco teórico se presenta la historia, métodos, tipologías y componentes de las impresoras 3D; historia y principales normativas asociadas a la gestión de activos y el mantenimiento, así como metodologías particulares como el Lean Manufacturing y sus herramientas; metodologías ágiles de diseño como Scrum y Design Thinking; herramientas y lenguajes de programación relativas al desarrollo de softwares o aplicaciones web. A continuación, se presenta el estado del arte respecto a trabajos o proyectos similares a esta memoria, como lo son softwares GMAO, gestión de la impresión 3D y el Design Thinking enfocado en software y aplicaciones web. La sección del desarrollo del proyecto se enfoca en la descripción de la empresa donde se desarrolla este trabajo, el problema a solucionar y la aplicación de la metodología de Design Thinking para el desarrollo de esta memoria. Luego, se muestran los resultados respecto a las pruebas realizadas, la definición de indicadores relevantes para el proceso, la toma de métricas y elaboración de documentación para instalación y uso de la aplicación web. Finalmente, se presentan las conclusiones y observaciones relativas a la realización de este trabajo, y la bibliografía utilizada. 

\clearpage

\section{Motivación}

El desarrollo de este trabajo responde a la participación del autor por más de dos años en empresas enfocadas al diseño 3D y la fabricación digital a través de la impresión 3D FDM. Dentro de las motivaciones para la realización de este trabajo, y basada en la experiencia práctica, se identifican:

\begin{enumerate}
	\item Plantear una solución tecnológica y eficiente para el control remoto de varias impresoras 3D con la menor cantidad de recursos posibles.
	\item Utilizar la información entregada por un software existente para ampliar las funciones que un operario realizar.
	\item Incorporar conocimientos de programación y control de versiones relativos a la disciplina de la Ingeniería Mecánica y el Diseño.
	\item Conocer y utilizar metodologías innovadoras para la gestión de activos y el mantenimiento.
	\item Desarrollar desde cero una aplicación web que interactúe con máquinas de impresión 3D y sea capaz de emitir, recibir, editar y borrar información.   
\end{enumerate} 





